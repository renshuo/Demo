\documentclass[]{beamer}
%\usetheme{Boadilla}
%\usetheme{Goettingen}
%\usetheme{Montpellier}
%\usetheme{metropolis}

%% \usetheme Antibes, Bergen, Berkeley, Berlin, Boadilla, boxes, Copenhagen, Darmstadt, default, Dresden, Frankfurt, Goettingen, Hannover, Ilmenau, JuanLesPins, Luebeck, Madrid, Malmoe, Marburg, Montpellier, PaloAlto, Pittsburgh, Rochester, Singapore, Szeged, Warsaw
%% \usecolortheme albatross, beetle, crane, default, dolphin, dove, fly, lily, orchid, rose, seagull, seahorse, sidebartab, whale
%% \usefonttheme default, professionalfonts, serif, structurebold, structureitalicserif, structuresmallcapsserif
%% \useinnertheme circles, default, inmargin, rectangles, rounded
%% \useoutertheme default, infolines, miniframes, shadow, sidebar, smoothbars, smoothtree, split,tree

% config chinese lang
\usepackage{xeCJK}
%  config font
\usepackage{fontspec}
\setCJKmainfont{微软雅黑}
% useable fonts: 宋体,微软雅黑


\begin{document}

\begin{frame}
  \title{幻灯测试}
  \author{sren}
  \date{\today}
  \maketitle
\end{frame}


\begin{frame}
  \normalsize
  \tableofcontents
\end{frame}

%\frame{\tableofcontents}




\section{第二段}
\frame{\tableofcontents[currentsection]}
\begin{frame}{第二段frame}
\frametitle[第二段的主题]{第二个主题}
  第二段frame的内容, latex: \TeX
\end{frame}

\section{第三段}
\frame{\tableofcontents[currentsection]}
\begin{frame}{第三段}
  第三端内容,其实这个是第\thesection 段。
\end{frame}

\section{第一段}
\subsection{第一个子段}
\frame{\tableofcontents[currentsection]}
\begin{frame}{自如之理,乃见真实}
  \pause % 用于点击继续功能
  \begin{block}{佛告须菩提} 
    凡所有相,皆是虚妄。若见诸相非相,则见如来。 
  \end{block} 
  \pause
  \begin{alertblock}{佛告须菩提} 凡所有相,皆是虚妄。若见诸相非相,则见如来。 
  \end{alertblock} 
  \pause
  \begin {exampleblock}{佛告须菩提} 凡所有相,皆是虚妄。若见诸相非相,则见如来。 
  \end{exampleblock}
\end{frame}

\subsection{第一段的第二个子段}

\end{document}
