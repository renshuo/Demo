\documentclass[UTF8]{ctexbeamer}
\usepackage{listings}
\usepackage{fontspec}
\usepackage{xcolor}
\usepackage{url}
\usepackage{hyperref}

%% listings config
\lstset{
  columns=fixed,
  language=java,
  numbers=left,
  numberstyle=\tiny,
  basicstyle=\small
}

\begin{document}

\begin{frame}
  \title{kotlin入门基础} \author{sren} \date{\today}
  \maketitle
\end{frame}

\begin{frame}
  \normalsize
  \tableofcontents
\end{frame}


\section{kotlin简介}
\frame{\tableofcontents[currentsection]} % 在每个章节前显示当前章节在目录中的位置

\begin{frame}{kotlin起源}
  kotlin是JetBrains公司设计的基于Jvm的静态编程语言,目的是成为更好的Java,在
  kotlin中改进Java语言的一些问题。
  kotlin编译的输出是Jvm字节码,在Jvm上运行。kotlin也可以编译为Javascript在浏览器
  中运行,或者编译为机器码在操作系统中运行。\\
  \vspace{5em}
  \zihao{-5}JetBrains: JetBrains成立于2000年,是一家捷克的软件公司,是 IntelliJ IDEA等一系列IDE的开发商。
\end{frame}

\begin{frame}{kotlin的现状}
  \begin{itemize}
  \item 一个“更好的Java” 
  \item Google将kotlin做为Andorid的官方语言
  \end{itemize}
\end{frame}

\begin{frame}{kotlin特性}
  \begin{itemize}
  \item 支持与Java混合编程
  \item 可以编译为JavaScript或机器码,支持在浏览器,Ios,MacOS等各种环境下运行
  \item 支持协程
  \item 大量的语法糖
  \item 函数式编程元素
  \end{itemize}
\end{frame}


\section{kotlin基本语法}
\frame{\tableofcontents[currentsection]}
\begin{frame}[fragile]{Hello World}
\begin{lstlisting}
package my.demo

import kotlin.text.*

fun main() {
    println(“hello world”)
}
\end{lstlisting}
\end{frame}

\begin{frame}[fragile]{类,对象}
\begin{lstlisting}
class Abc(val va1: String, var va2: Int) : Zyx() {

  val constV1 = 0
  init {
    // first init
  }
  
  var variable2 = 1
  init {
    // second init
  }

  late init var vari3
  
  constructor() {
    // 
  }
}

val abc: Abc = Abc()
\end{lstlisting}
\end{frame}

\begin{frame}[fragile]{类函数}
\begin{lstlisting}
class Abc{
  private fun func1(in1: String = "default", in2 Int): String {
     val str:String = this.func1("abc", 2)   
  }
  
  fun <T> func2(in1: T, dts: Int): List<T> {
      val res: List<Int> = this.func2(0, 13)
  }

  fun func3(in1: Int=0, cls: (it:String)-> Unit) {
     val r1: Int = this.func3(0,{it:String -> ... })
     val r2: Int = this.func3(0){it:String -> ... }
  }
}
\end{lstlisting}
\end{frame}


\begin{frame}[fragile]{数据类}
\begin{lstlisting}
data class Person(val name:String){
  var age = 0
}
\end{lstlisting}
  相当于Java的:
\begin{lstlisting}
class Person{
  String name;
  Int age;

  public String getName(){ return this.name}
  public void setName(String name){ this.name = name}
  public Int getAge(){ return this.age}
  public void setAge(Int age){ this.age = age}

  public String toString(){
    return StringBuilder.build().apppend("Person {")...
  }
}
\end{lstlisting}
\end{frame}

\begin{frame}[fragile]{对象}
\begin{lstlisting}
val a = object {
  var x: Int = 0
  var y: Int = 1
}
print(a.x + a.y)
\end{lstlisting}
\end{frame}


\begin{frame}[fragile]{Lambda表达式}
定义一个lambda表达式
\begin{lstlisting}
 val sum: (Int, Int) -> Int = {x:Int, y:Int -> x+y }

 val sum = {x, y -> x+y }
\end{lstlisting}
 \vspace{40pt}
调用Lambda,将Lambda表达式作为参数传递
\begin{lstlisting}
 val headSize = 2
 val result = list.map { it ->
   it.subString(headSize)
 }
\end{lstlisting}
\end{frame}


\section{重要特性}
\frame{\tableofcontents[currentsection]}
\begin{frame}[fragile]{空安全}
  kotlin空安全是对Java中的NPE(NullPointerException)的一个解决方案,其做法是限制
  null值的产生,且默认情况下,变量必然不为空
  \\
  相关操作:
\begin{lstlisting}
var a: String ="abc"
a = null // error

var b: String? ="abc"
b = null // ok

println(b?.length) //if b==null then null else b.length
println(b?.length ?: -1) // if b==null then -1 else b.length

println(b!!.length) // if b==null then throw NPE
\end{lstlisting}
\end{frame}

\begin{frame}[fragile]{解构}
\begin{lstlisting}
data class A(x: Int, y: Int)

fun test(): A {}

val (x, y) = test()
val (_, y) = test()
\end{lstlisting}
\end{frame}

\begin{frame}[fragile]{扩展}
  kotlin可以对一个类的属性和方法进行扩展,且不需要继承:
\begin{lstlisting}
fun String.printLog(){
  print(this.value)
}
\end{lstlisting}
  扩展String类,增加printLog方法。
\end{frame}


\begin{frame}[fragile]{集合操作}
\begin{lstlisting}
val list = listof(1, 2, 3)
val l2 = list.map { it + 1 }
val l1 = list.filter { it > 1 }
val l3 = lsit.take(2)
val l4 = list.group { it }
val l5 = list.sortedBy { 3-it}
val l6 = list.sumBy { it * 2}
val l7 = list.reduce { sum, item -> sum + item }
\end{lstlisting}
\end{frame}

% \begin{frame}[fragile]{协程}
%   轻量级的线程\\
%   非常高的执行效率,不需要线程锁,没有资源冲突
  
  
% \begin{lstlisting}
% import kotlinx.coroutines
% \end{lstlisting}

%   参考资料:
%   \url{https://www.kotlincn.net/docs/reference/coroutines/coroutines-guide.html}
%   % \hyperref[kkotl]{https://www.kotlincn.net/docs/reference/coroutines/coroutines-guide.html}

% \end{frame}

\section{工程管理}
\frame{\tableofcontents[currentsection]}
\begin{frame}[fragile]{利用gradle管理项目}
  gradle是基于apache ant和maven概念的项目自动化构建工具,gradle基于groovy语言编
  写。也支持基于kotlin编写构建脚本。目前支持Java,Groovy,kotlin,scala等Jvm语言。\\
  \vspace{1em}
  gradle构建内容:
  \begin{enumerate}
  \item 源代码管理
  \item 依赖管理
  \item 编译
  \item 测试
  \item 打包
  \end{enumerate}
  % \hyperref[build.gradle]{./resource/build.gradle}
  \href{./resource/build.gradle}{bu
    ild.grale}
\end{frame}


% \begin{frame}[fragile]{a}
% \begin{lstlisting}
% \end{lstlisting}
% \end{frame}

\end{document}
