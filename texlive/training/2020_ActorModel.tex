\documentclass[UTF8]{ctexbeamer}
\usepackage{listings}
\usepackage{fontspec}
\usepackage{xcolor}
\usepackage{url}
\usepackage{hyperref}
\usepackage{tikz}
\usetikzlibrary{shapes,arrows}

%% listings config
\lstset{
  columns=fixed,
  language=java,
  numbers=left,
  numberstyle=\tiny,
  basicstyle=\small
}

\begin{document}

\begin{frame}
  \title{Actors编程模型} \author{sren} \date{\today}
  \maketitle
\end{frame}

\begin{frame}
  \begin{center}目录\end{center}
  \normalsize
  \tableofcontents
\end{frame}

\section{Actors基本概念}
\frame{\tableofcontents[currentsection]} % 在每个章节前显示当前章节在目录中的位置

\begin{frame}{actor模型}
  Actor模型是一种并发模型,通过组件方式定义并发编程范式的高级阶段,避免使用者接
  触多线程并发或线程池等基础概念,简化并发程序的开发难度。\\
\end{frame}

\begin{frame}{actor模型}
  \begin{center}
    \tikzstyle{actor} = [circle, draw, fill=blue!20, text centered]
    \tikzstyle{msg} = [rectangle, draw, fill=blue!60, text centered]
    \tikzstyle{line} = [draw, -latex']

    \begin{tikzpicture}[node distance=2cm, auto]
      \node[msg] (m1) {Mail Box};
      \node[actor, below of=m1] (A1) {Actor1};
      \node[msg, right of=m1, node distance=4cm] (m2) {Mail Box};
      \node[actor, below of=m2] (A2) {Actor2};
      \node[actor, above of=m1, right=1.25cm] (a3) {Actor3};
      \node[msg, above of=a3] (m3) {Mail Box};

      \path [line] (m1) -> (A1);
      \path [line] (A1) -- (-1.5,-2) -- (-1.5, 4) ->  (m3);
      \path [line] (m2) -> (A2);
      \path [line] (A2) -> (5.5, -2) -- (5.5, 4) ->  (m3);
      \path [line] (m3) -> (a3);
      \path [line] (a3) -> (m1);
      \path [line] (a3) -> (m2);
    \end{tikzpicture}
  \end{center}
\end{frame}

\begin{frame}{actor模型}
  基本概念:
  \begin{itemize}
  \item actor:\\
    actor是一个轻量级的实体,用来接收事件并对事件进行处理。actor可以是有状态或无状态的。
  \item message:\\
    message是actor之间传递的信息/命令/事件等,在actor模型中,message是不可变的。
  \item mailbox:\\
    mailbox是用来存储消息的队列,actor会从mailbox获取消息并处理。
  \end{itemize}
\end{frame}

\begin{frame}{actor模型}
  基本概念: 
  \begin{itemize}
  \item 并行/并发 \\
    并行是系统同时执行多个任务,并发是系统分时执行多个任务
  \item 异步/同步 \\
    同步是调用者需要等待返回,期间不做任何事情。
    异步是调用者不等待返回,继续做其他事情,当有返回信息时再具体处理。
  \end{itemize}
\end{frame}



\end{document}