\documentclass[a4paper,12pt]{article}


\usepackage{xeCJK}
\setCJKmainfont{微软雅黑}

%opening
\title{scrum实践}
\author{sren}
\date{\today}

\begin{document}

\maketitle

\newpage

\tableofcontents

\newpage

% ------------------------------------
\begin{quote}
 \section{backlog/story}

 一个故事(story)或者叫条目(backlog),要有以下元素:
 
 \begin{enumerate}
  \item ID: 统一标识符
  \item Name: 名称,简短的描述性的故事名。
  \item Importance: 重要性
%   \item Initial estimate: 初始估算的工作量,最小单位是故事点(story point),大致相当于一个理想的人天。
%   \item How to Demo: 如何做演示,本质是一个简单的测试方法描述
  \item notes: 注解,相关的其他信息。
 \end{enumerate}

一个产品backlog实例:
\begin{table}
\begin{tabular}{|l|c|c|c|c|r|}
\hline
1	&存款	&30	&5	&登录,打开存款界面,存入10元,转到我的账户余额,检查余额增加了10元	&需要UML顺序图	\\
\hline
2	&查看自己的交易明细 &10 &5 &登录,点击交易,存钱,返回交易页面,看到新的存款显示。 &使用分页技术避免大规模数据库查询 \\
\hline
\end{tabular}
\caption{backlog实例}
\label{图表1}
\end{table}


\section{怎样准备sprint计划}
\textbf{在sprint计划前,要确保产品backlog井然有序。}

\begin{enumerate}
  \item 产品backlog必须存在
  \item 只能有一个backlog和一个产品负责人
  \item 所有重要的backlog条目都已经根据重要性被评过分,不同的重要程度对应不同的分值。不要分值连续,留出空间。
    \begin{itemize}
     \item 只要确认在下一个sprint中出现,则一定要划分到一个特有的重要层次
     \item 分数只代表重要性,相对性不与分值相比较
    \end{itemize}
  \item 产品负责人必须理解所有条目的含义。
 \end{enumerate}

\section{服务端}
 
\section{客户端}

\end{quote}
\end{document}
