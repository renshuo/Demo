\documentclass[a4paper,12pt]{article} % 使用a4纸,12pt字体
\usepackage{fullpage} % use 1 inch margins
\usepackage{tabularx}
\usepackage{booktabs} % 提供 \toprule, midrule, bottomrule 用于表格横线
\usepackage{graphicx}
\usepackage{xecolor}
\usepackage{hyperref}
\usepackage{xeCJKfntef} % 提供下划线,删除线等特效
%\textwidth=350pt \textheight=22cm

% config chinese lang
\usepackage{xeCJK}
%  config font
%\usepackage{fontspec}
\setCJKmainfont{微软雅黑}
% useable fonts: 宋体,微软雅黑
{\newCJKfontfamily\kai{楷体_GB2312}}
{\newCJKfontfamily\song{宋体}}
{\newCJKfontfamily\hei{黑体}}
{\newCJKfontfamily\yahei{微软雅黑}}
{\newCJKfontfamily\zhei{文泉驿正黑}}



% 中文化配置
\renewcommand{\tablename}{表表}
\renewcommand{\figurename}{图图}
\renewcommand{\contentsname}{\centering 目录录} % 将英文的contents改为目录,并居中
\renewcommand{\listtablename}{\centering 表目录}
\renewcommand{\listfigurename}{\centering 图目录}
\renewcommand{\abstractname}{\centering 摘要} 

%%%%%%%%%%% finsih config %%%%%%%%%%%%%%%%%%%%%%%%%%%%%%%%%%%%%%%%%%%%%%%%%%%%%%%%%

%标题首页
\title{LaTeX样例文档}
\author{sren}
\date{\today}

\begin{document}
\maketitle
\thispagestyle{empty} % 将本页的style设置为空,页码不会出现


% 摘要
\newpage
\pagenumbering{Roman}
摘要部分


% 目录
\newpage
\tableofcontents
\newpage
\listoffigures
\newpage
\listoftables

\newpage % 新一页
\pagenumbering{arabic}
%\setcounter{page}{1} 设置这里页码从1重新开始
正文开始
 \section{概述}
 本文是latex的文档样例文件,大多数常用的latex命令用法\\在这里都应该有。
 如果没有,那就是你不需要的!
  % \\是换行,空行则表示下一段落,段落头会有indent

 概述页的第二段说明,本段没啥说的。注意本段有intent。

 {\Huge 概述第三段,本段测试下自动换行,即超过一行长度的文本,看latex怎么处理。为了方便超过一行,所以使用了Huge级别的字。}
 {\kill 本行不打印}
 \pagebreak

 
 \section[pz字体]{配置字体} % pz字体 是出现在目录中的文字,配置字体 是这个位置显示的文字
 \subsection{不同的字体}
 字体族需要事先定义,定义的时候会有对应的命令。\\
 {\song 一些宋体字}\\ % 双反斜杠结束本行
 {\yahei 一些微软雅黑字体}\\
 {\hei 一些黑体字}\\
 {\zhei 一些文泉驿正黑体字}\\
 %% \setCJKfamilyfont{xxx}{楷体_GB2312}
 %% {\CJKfamily{xxx} 这是应该是楷体了吧}\\
 {\kai 这个应该是楷体了}\\
 {\CJKfontspec{方正姚体_GBK} 这个字体是本地定义的}

 \subsection*{不同的字大小} % * 表示不进入section编号,也不出现在目录中
 {\tiny 最小的字 }\\
 {\scriptsize 第二小的字,脚本用的字大小}\\
 {\footnotesize 第三小的字,脚标用的}\\
 {\small 小字}\\
 {\normalsize 正常大小的字,应该就是正文用的字 }\\
 {\large 大的字 }\\
 {\Large 更大一点的}\\
 {\LARGE 再大一点}\\
 {\huge 巨大的字}\\
 {\Huge 更巨大的字}\\* % 结束本行,禁止分页


 \section{引用}
 quote引用用来引用别的文字段落。quote整体缩进,但段首不缩进,
 \begin{quote}
   public void static main(String[] args){
     System.out.println("hello world");
   }
  \end{quote}
 
 verbatim用于贴代码,或者其他格式文本,verbatim可以保持文本的原始格式。
\begin{verbatim}
  public static void main(String[] args){
    System.out.println("hello world!");
  }
\end{verbatim}
 简化版的verbatim即verb指令用法是这样的\verb-\verb+abc+-。其中的加号可以替换成别的,比如减号。
 实际上,verbatim环境是用+替代了\verb-\-,用\verb-\[-替代了\verb-{-,用\verb-\]-替代了\verb-}-,所以,在verb环境下如果想要排latex的话, 就用这3个符号就行。


 \section{文字块}
 \mbox{文字块是一段在块里的文字\\但是没有外框}\\
 \fbox{这个是有外框的文字块}\\
 \makebox[14cm][r]{makebox是mbox的原型?,可以加参数:右排版,宽14cm}\\
 \framebox[14cm][s]{s表示\hfill 均匀分布}\\
 一个方块
 \framebox{\rule{3mm}{0pt}\rule{0pt}{3mm}}\\
 \newsavebox{\savbox}
 \sbox{\savbox}{\framebox{储存方块,用于存东西,可以在别的地方重复引用}}
 \usebox{\savbox}, \usebox{\savbox}, \usebox{\savbox}\\

 
 \section{列表样例}
 编号的列表
 \begin{enumerate}
  \item 第一子系统\\
  第一子系统的说明。
  \item 第二子系统\\
  第二子系统的说明。
 \end{enumerate}

 不编号的列表
 \begin{itemize} %不编号列表
 \item lsit 1
 \item list 2
 \end{itemize}
 

\section{图像}

\subsection{插入图像文件}
插入一幅本地图片:\\
\begin{figure}[htbp] % h: here, t: top, b: bottom, p: float page
  \centering
  \includegraphics[width=10cm,height=15cm,keepaspectratio]{pic1.jpg}
  \caption{看流星}
  \label{fig:star1}
\end{figure}

\subsection{关于色彩} % TODO
%% \textcolor {Red}{红}
%% \textcolor{Green}{绿}
%% \textcolor{Blue}{蓝}
%% \textcolor[RGB]{255 ,0 ,0}{红}
%% \textcolor[HTML]{00 FF 00}{绿}
%% \textcolor[rgb]{0 ,0 ,1}{蓝}

\subsection{矢量绘图}
文档上介绍的几个太复杂了,先用dot吧\\


\section{表格}
表格前的正文说明
\begin{table}[!th]
\begin{tabular}{|l|c|c|c|c|r|}
\toprule
1	&存款	&30	&5	&登录,打开存款界面,存入10元,转到我的账户余额,检查余额增加了10元	&需要UML顺序图	\\
\midrule
2	&查看自己的交易明细 &10 &5 &登录,点击交易,存钱,返回交易页面,看到新的存款显示。 &使用分页技术避免大规模数据库查询 \\
\bottomrule
\end{tabular}
\caption{表格实例}
\label{ex:table}
\end{table}
表格后的正文说明,引用表格:\ref{ex:table}。

{带有表线,指定宽度的表格}
\begin{table}[htbp]
  \centering
  \begin{tabular}{|p{80pt}|>{\centering}p{80pt}|>{\raggedleft\arraybackslash}p{80pt}|}
    \toprule
    x1  & x2  & x3\\
    \midrule
    x1  & x2  & x3\\
    x1  & x2  & x3\\
    \bottomrule
  \end{tabular}
\end{table}


\section{超链接}
\url{http://192.169.1.254/bbs}\\
\href{http://192.169.1.254/bbs}{这是我们的BBS}


\section{页眉页脚} %TODO


\section{xeCJK}
\subsection{汉字基本特效}
\CJKunderline{下划线}\\
\CJKunderdblline{双下划线}
\CJKunderwave{下波浪线}
\CJKsout{删除线abc}
\CJKxout{斜删除线abc}
\CJKunderdot{下点符号}
\CJKunderanysymbol{0.2em}{\tiny$\triangle$}{将任意符号放在文字下}\\


\end{document}
