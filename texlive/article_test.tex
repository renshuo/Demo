\documentclass[a4paper,12pt]{article} % 使用a4纸,12pt字体
\usepackage{fullpage} % use 1 inch margins
%%% fullpage 和 fancyhdr有冲突,会导致第一个有页眉的页面中,正文和页眉重叠
%%% 另外,fullpage也会使其余页面的页眉离正文过近(几乎要重叠)
\usepackage{tabularx}
\usepackage{booktabs} % 提供 \toprule, midrule, bottomrule 用于表格横线
\usepackage{graphicx} % 插入图片
\usepackage{xecolor} % 色彩
\usepackage{hyperref} % 超链接
\usepackage{xeCJKfntef} % 提供下划线,删除线等特效
\usepackage{indentfirst} %提供中文首行缩进
\usepackage{bbding} % 一些有意思的符号
\usepackage{tikz}
\usepackage{pgf-umlsd} % 画UML图
\usepgflibrary{arrows} % pgf-umlsd需要
\usepackage{listings} % 引用编程语言
\usepackage{fancyhdr} % 管理页眉页脚
\usepackage{flowchart} % 绘制流程图
\usetikzlibrary{arrows} % flowchart需要
\usepackage{float} % 提供固定位置表格(即非浮动表格/图像功能) 在表格中使用 H 表示强制here

%\textwidth=350pt \textheight=22cm

\usepackage{xeCJK}
\setCJKmainfont{微软雅黑}
{\newCJKfontfamily\kai{楷体_GB2312}}
{\newCJKfontfamily\song{宋体}}
{\newCJKfontfamily\hei{黑体}}
{\newCJKfontfamily\yahei{微软雅黑}}
{\newCJKfontfamily\zhei{文泉驿正黑}}


% 中文化配置
\renewcommand{\tablename}{表表}
\renewcommand{\figurename}{图图}
\renewcommand{\contentsname}{\centering 目录录} % 将英文的contents改为目录,并居中
\renewcommand{\listtablename}{\centering 表目录}
\renewcommand{\listfigurename}{\centering 图目录}
\renewcommand{\abstractname}{\centering 摘要} 

%%%%%%%%%%% finsih config %%%%%%%%%%%%%%%%%%%%%%%%%%%%%%%%%%%%%%%%%%%%%%%%%%%%%%%%%

%标题首页
\title{LaTeX样例文档}
\author{sren}
\date{\today}

\begin{document}
\maketitle
\thispagestyle{empty} % 将本页的style设置为空,页码不会出现


% 摘要
\newpage
\pagenumbering{Roman}
摘要部分


% 目录
\newpage
\tableofcontents                
\newpage
\listoffigures
\newpage
\listoftables


\newpage % 新一页
\pagenumbering{arabic}
%% 页眉页脚 %% 一般,页眉从正文首页开始,目录部分没有页眉,只有页脚的罗马数字页码
\pagestyle{fancy}
\newsavebox{\headpic}
\sbox{\headpic}{\includegraphics{headLogo.png}}
\lhead{}
\chead{\usebox{\headpic}民用遥感卫星海军应用系统数据存储管理分系统详细设计}
\rhead{}
\lfoot{中国航天科技集团公司五院西安分院}
\cfoot{}
\rfoot{第\thepage 页}
\renewcommand{\headrulewidth}{0.4pt}
\renewcommand{\footrulewidth}{0.4pt}
%% end


%\setcounter{page}{1} 设置这里页码从1重新开始
 正文开始
 \section{概述}
 本文是latex的文档样例文件,大多数常用的latex命令用法\\在这里都应该有。
 如果没有,那就是你不需要的!
  % \\是换行,空行则表示下一段落,段落头会有indent

  概述页的第二段说明,本段没啥说的。注意本段有intent。

  {\Huge 概述第三段,本段测试下自动换行,即超过一行长度的文本,看latex怎么处理。为了方便超过一行,所以使用了Huge级别的字。}
% {\kill 本行不打印 } %% \kill 和tikz有冲突
  \pagebreak

 
 \section[pz字体]{配置字体} % pz字体 是出现在目录中的文字,配置字体 是这个位置显示的文字
 \subsection{不同的字体}
 字体族需要事先定义,定义的时候会有对应的命令。\\
 {\song 一些宋体字}\\ % 双反斜杠结束本行
 {\yahei 一些微软雅黑字体}\\
 {\hei 一些黑体字}\\
 {\zhei 一些文泉驿正黑体字}\\
 %% \setCJKfamilyfont{xxx}{楷体_GB2312}
 %% {\CJKfamily{xxx} 这是应该是楷体了吧}\\
 {\kai 这个应该是楷体了}\\
 {\CJKfontspec{方正姚体_GBK} 这个字体是本地定义的}

 \subsection*{不同的字大小} % * 表示不进入section编号,也不出现在目录中
 {\tiny 最小的字 }\\
 {\scriptsize 第二小的字,脚本用的字大小}\\
 {\footnotesize 第三小的字,脚标用的}\\
 {\small 小字}\\
 {\normalsize 正常大小的字,应该就是正文用的字 }\\
 {\large 大的字 }\\
 {\Large 更大一点的}\\
 {\LARGE 再大一点}\\
 {\huge 巨大的字}\\
 {\Huge 更巨大的字}\\* % 结束本行,禁止分页


 \section{引用}
 quote引用用来引用别的文字段落。quote整体缩进,但段首不缩进,
 \begin{quote}
   public void static main(String[] args){
     System.out.println("hello world");
   }
  \end{quote}
 
 verbatim用于贴代码,或者其他格式文本,verbatim可以保持文本的原始格式。
\begin{verbatim}
  public static void main(String[] args){
    System.out.println("hello world!");
  }
\end{verbatim}
 简化版的verbatim即verb指令用法是这样的\verb-\verb+abc+-。其中的加号可以替换成别的,比如减号。
 实际上,verbatim环境是用+替代了\verb-\-,用\verb-\[-替代了\verb-{-,用\verb-\]-替代了\verb-}-,所以,在verb环境下如果想要排latex的话, 就用这3个符号就行。


 \section{文字块}
 \mbox{文字块是一段在块里的文字\\但是没有外框}\\
 \fbox{这个是有外框的文字块}\\
 \makebox[14cm][r]{makebox是mbox的原型?,可以加参数:右排版,宽14cm}\\
 \framebox[14cm][s]{s表示\hfill 均匀分布}\\
 一个方块
 \framebox{\rule{3mm}{0pt}\rule{0pt}{3mm}}\\
 \newsavebox{\savbox}
 \sbox{\savbox}{\framebox{储存方块,用于存东西,可以在别的地方重复引用}}
 \usebox{\savbox}, \usebox{\savbox}, \usebox{\savbox}\\

 
 \section{列表样例}
 编号的列表
 \begin{enumerate}
  \item 第一子系统\\
  第一子系统的说明。
  \item 第二子系统\\
  第二子系统的说明。
 \end{enumerate}

 不编号的列表
 \begin{itemize} %不编号列表
 \item lsit 1
 \item list 2
 \end{itemize}

 \subsection{嵌套列表}
 嵌套列表
 \begin{enumerate}
 \item 主列表第一项
 \item 主列表第二项
   \begin{enumerate}
   \item 子列表第一项
   \item 子列表第二项
   \end{enumerate}
   \item 主列表剩余项
 \end{enumerate}
 

\section{图像}

\subsection{插入图像文件}
插入一幅本地图片:\\
\begin{figure}[htbp] % h: here, t: top, b: bottom, p: float page
  \centering
  \includegraphics[width=10cm,height=15cm,keepaspectratio]{pic1.jpg}
  \caption{看流星}
  \label{fig:star1}
\end{figure}

\subsection{关于色彩} % TODO
%% \textcolor {Red}{红}
%% \textcolor{Green}{绿}
%% \textcolor{Blue}{蓝}
%% \textcolor[RGB]{255 ,0 ,0}{红}
%% \textcolor[HTML]{00 FF 00}{绿}
%% \textcolor[rgb]{0 ,0 ,1}{蓝}

\subsection{矢量绘图}
文档上介绍的几个太复杂了,先用dot吧\\
\subsection{UML绘图}
\subsubsection{时序图}
\begin{figure}
  \centering
\begin{sequencediagram}
  \newthread{ss}{SimulationServer}
    \newinst[2]{ctr}{SimControlNode}
    \newinst[2]{ps}{PhysicsServer}
    \newinst[2]{sense}{SenseServer}

    \begin{call}{ss}{Initialize()}{sense}{}
    \end{call}
    \begin{sdblock}{Run Loop}{in a block}
      \begin{call}{ss}{StartCycle()}{ctr}{}
        \begin{call}{ctr}{ActAgent()}{sense}{}
        \end{call}
      \end{call}
      \begin{call}{ss}{Update()}{ps}{}
        \begin{call}{ps}{PrePhysicsUpdate()}{sense}{state}
        \end{call}
        \begin{callself}{ps}{PhysicsUpdate()}{}
        \end{callself}
        \begin{call}{ps}{PostPhysicsUpdate()}{sense}{}
        \end{call}
      \end{call}
      \begin{call}{ss}{EndCycle()}{ctr}{}
        \begin{call}{ctr}{SenseAgent()}{sense}{}
        \end{call}
      \end{call}
     \end{sdblock}
\end{sequencediagram}
\caption{UML test}
\end{figure}

\subsubsection{流程图}
\begin{figure}[H]
  \centering
  \begin{tikzpicture}
    \node (start) at (0,0) [draw, terminal] {开始};
    \node (task) at (0,-1) [draw, predproc] {收到任务请求};
    \node (listen) at (0,-2) [draw, process] {开启socket};
    \node (receive) at (0,-3) [draw, predproc] {接收业务数据};
    \coordinate (pt1) at (3,-5) ;
    \node (check) at (0,-4) [draw, process] {规范化检验};
    \node (draw) at (0,-5) [draw, process] {提取业务信息};
    \node (index) at (0,-6) [draw, process] {生成编目信息};
    \node (savedb) at (0,-7) [draw, process] {编目信息入库};
    \node (saveyun) at (0,-8) [draw, process] {数据文件存储归档};
    \node (finish) at (0,-9) [draw, predproc] {收到任务结束信息};
    \node (end) at (0,-10) [draw, terminal] {结束};
    \draw[->] (start) -- (task);
    \draw[->] (task) -- (listen);
    \draw[->] (listen) -- (receive);
    \draw[->] (receive) -- (check);
    \draw[->] (check) -- (draw);
    \draw[->] (draw) -- (index);
    \draw[->] (index) -- (savedb);
    \draw[->] (savedb) -- (saveyun);
    \draw[->] (saveyun) -| (pt1) |- (receive);
    \draw[->] (saveyun) -- (finish);
    \draw[->] (finish) -- (end);
    \end{tikzpicture}
\caption{数据编目软件执行流程图}
\label{fig:dp_timer}
\end{figure}

\section{表格}
表格前的正文说明
\begin{table}[!th]
\begin{tabular}{|l|c|c|c|c|r|}
\toprule
1	&存款	&30	&5	&登录,打开存款界面,存入10元,转到我的账户余额,检查余额增加了10元	&需要UML顺序图	\\
\midrule
2	&查看自己的交易明细 &10 &5 &登录,点击交易,存钱,返回交易页面,看到新的存款显示。 &使用分页技术避免大规模数据库查询 \\
\bottomrule
\end{tabular}
\caption{表格实例}
\label{ex:table}
\end{table}
表格后的正文说明,引用表格:\ref{ex:table}。

{带有表线,指定宽度的表格}
\begin{table}[htbp]
  \centering
  \begin{tabular}{|p{80pt}|>{\centering}p{80pt}|>{\raggedleft\arraybackslash}p{80pt}|}
    \toprule
    x1  & x2  & x3\\
    \midrule
    x1  & x2  & x3\\
    x1  & x2  & x3\\
    \bottomrule
  \end{tabular}
\end{table}


\section{超链接}
\url{http://192.169.1.254/bbs}\\
\href{http://192.169.1.254/bbs}{这是我们的BBS}


\section{xeCJK}
\subsection{汉字基本特效}
\CJKunderline{下划线}\\
\CJKunderdblline{双下划线}
\CJKunderwave{下波浪线}
\CJKsout{删除线abc}
\CJKxout{斜删除线abc}
\CJKunderdot{下点符号}
\CJKunderanysymbol{0.2em}{\tiny$\triangle$}{将任意符号放在文字下}\\


\section{引用编程语言}
% 引用一段java代码,但是只输出2到5行,左侧显示行号,隔几个号码显示,距离代码框10pt
\begin{lstlisting}[language=Java,firstline=2,lastline=5,numbers=left,numberstyle=\large,stepnumber=2,numbersep=10pt]
  import java.util.Map;
  public class ABC{
  public static void main(String[] args){
    System.out.println(``hello world'');
  }
  }
\end{lstlisting}
  
\section{bbding,一些有意思的符号}
\begin{Huge}
\HandRight \HandLeft \XSolid \Plus \Cross \CrossClowerTips \\
\CrossMaltese \FiveStar \FiveStarLines \SixStar \EightStar \TwelweStar \\
\SixteenStarLight \FiveFlowerOpen \FiveFlowerPetal \Snowflake \SnowflakeChevron \Sparkle \\
\SquareSolid \Square \TriangleUp \TriangleDown \OrnamentDiamondSolid \Ellipse \\
\Phone \Tape \Plane \Envelope \Peace \Checkmark \\
\end{Huge}

\end{document}
